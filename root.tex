%% GABARIT POUR MÉMOIRE STANDARD
%%
%% Consulter la documentation de la classe ulthese pour une
%% description détaillée de la classe, de ce gabarit et des options
%% disponibles.
%%
%% [Ne pas hésiter à supprimer les commentaires après les avoir lus.]
%%
%% Déclaration de la classe avec le type de grade
%%   [l'un de MSc, LLM, MA, MMus, MServSoc, MScGeogr, MATDR]
%% et les langues les plus courantes. Le français sera la langue par
%% défaut du document.
\documentclass[MSc,english,francais]{ulthese}
  %% Encodage utilisé pour les caractères accentués dans les fichiers
  %% source du document. Les gabarits sont encodés en UTF-8. Inutile avec
  %% XeLaTeX, qui gère Unicode nativement.
  \ifxetex\else \usepackage[utf8]{inputenc} \fi

  %% Charger ici les autres paquetages nécessaires pour le document.
  %% Quelques exemples; décommenter au besoin.
  %\usepackage{amsmath}          % recommandé pour les mathématiques
  %\usepackage{ncccomma}         % gestion de la virgule dans les nombres

  %% Utilisation d'une autre police de caractères pour le document.
  %% - Sous LaTeX
  %\usepackage{mathpazo}         % texte et mathématiques en Palatino
  %\usepackage{mathptmx}         % texte et mathématiques en Times
  %% - Sous XeLaTeX
  %\setmainfont{TeX Gyre Pagella}      % texte en Pagella (Palatino)
  %\setmathfont{TeX Gyre Pagella Math} % mathématiques en Pagella (Palatino)
  %\setmainfont{TeX Gyre Termes}       % texte en Termes (Times)
  %\setmathfont{TeX Gyre Termes Math}  % mathématiques en Termes (Times)

  %% Gestion des hyperliens dans le document. S'assurer que hyperref
  %% est le dernier paquetage chargé.
  \usepackage{hyperref}
  \hypersetup{colorlinks,allcolors=ULlinkcolor}

  %% Options de mise en forme du mode français de babel. Consulter la
  %% documentation du paquetage babel pour les options disponibles.
  %% Désactiver (effacer ou mettre en commentaire) si l'option
  %% 'nobabel' est spécifiée au chargement de la classe.
  \frenchbsetup{%
    CompactItemize=false,         % ne pas compacter les listes
    ThinSpaceInFrenchNumbers=true % espace fine dans les nombres
  }

  %% Style de la bibliographie.
  \bibliographystyle{}

  %% Déclarations des pages de titre. Remplacer les éléments entre < >.
  %% Supprimer les caractères < >. Couper un long titre ou un long
  %% sous-titre manuellement avec \\.
  \titre{<Titre principal>}
  % \titre{Ceci est un exemple de long titre \\
  %   avec saut de ligne manuel}
  % \soustitre{Sous-titre le cas échéant}
  % \soustitre{Ceci est un exemple de long sous-titre \\
  %   avec saut de ligne manuel}
  \auteur{<Prénom Nom>}
  \annee{<20xx>}
  \programme{Maîtrise en <discipline> <-- majeure, s'il y a lieu>}
  \direction{<Prénom Nom>, <directeur ou directrice> de recherche}
  % \codirection{<Prénom Nom>, <codirecteur ou codirectrice> de recherche}
  % \codirection{<Prénom Nom>, <codirecteur ou codirectrice> de recherche \\
  %              <Prénom Nom>, <codirecteur ou codirectrice> de recherche}

\begin{document}

\frontmatter                    % pages liminaires

\pagestitre                     % production des pages de titre

\chapter*{Résumé}                      % ne pas numéroter
\phantomsection\addcontentsline{toc}{chapter}{Résumé} % inclure dans TdM

\begin{otherlanguage*}{french}
  Texte du résumé en français.
\end{otherlanguage*}
                % résumé français
\chapter*{Abstract}                      % ne pas numéroter
\phantomsection\addcontentsline{toc}{chapter}{Abstract} % inclure dans TdM

\begin{otherlanguage*}{english}
  Text of English abstract.
\end{otherlanguage*}
              % résumé anglais
\cleardoublepage

\tableofcontents                % production de la TdM
\cleardoublepage

\listoftables                   % production de la liste des tableaux
\cleardoublepage

\listoffigures                  % production de la liste des figures
\cleardoublepage

\dedicace{Dédicace si désiré}
\cleardoublepage

\epigraphe{Texte de l'épigraphe}{Source ou auteur}
\cleardoublepage

\chapter*{Remerciements}         % ne pas numéroter
\phantomsection\addcontentsline{toc}{chapter}{Remerciements} % inclure dans TdM

Texte des remerciements en prose.
         % remerciements
\chapter*{Avant-propos}         % ne pas numéroter
\phantomsection\addcontentsline{toc}{chapter}{Avant-propos} % inclure dans TdM

L'avant-propos est surtout nécessaire pour une thèse par article.
           % avant-propos

\mainmatter                     % corps du document

\chapter*{Introduction}         % ne pas numéroter
\phantomsection\addcontentsline{toc}{chapter}{Introduction} % inclure dans TdM

Une thèse ou un mémoire devrait normalement débuter par une
introduction. Celle-ci est traitée comme un chapitre normal, sauf
qu'elle n'est pas numérotée.
          % introduction
\chapter{Description du robot 3 degrès de liberté}     % numéroté

Afin de réaliser l'implémentation de l'algoritme d'évitement de collisions sur un robot 3 degrès de liberté, une base robotique à du être sélectionnée.
Le robot utilisé est un robot Jaco, de l'entreprise Kinova, modifié pour les besoins du projet.
Le présent chapitre traite alors des différentes charactéristiques physiques du robot modifié, des modèles de cinématique directe et inverse ainsi que du calcul de la matrice Jacobienne du système.

\section{Description des charactéristiques physiques du robot}
Le robot Jaco, de Kinova, est un robot 6 degrès de liberté utilisé principalement dans le domaine de la santé où il est utilisé, entre autre, pour augmenter l'autonomie de patients ayant perdu le plein usage de leur bras.
Le robot, dans sa configuration normale, pèse 5.2 Kg, peut soulever une charge maximale de 1.6 Kg de manière continue et posède une porté de 90 cm.
L'achitecture du robot Jaco comprend 6 membrures creuses en fibre de carbone ainsi qu'un effecteur comprenant 3 doights ayant 2 articulations.

La figure \ref{fig:jaco_6dof} présente l'architecture du robot dans ça configuration originale.
\begin{figure}
 \begin{center}
  \begin{tabular}{c}
    \includegraphics[trim=0cm 0cm 0cm 0cm, scale=0.17]{"figures/jaco_6dof"}
  \end{tabular}
 \end{center}
\caption{Représentation de l'architecture du robot Jaco 6DOF}
 \label{fig:jaco_6dof}
\end{figure}

Deux modèles différents d'actuateurs sont présents dans le robot, soit le modèle K-75+ pour les 3 première articulations et le modèle K-58 pour les 3 autres articulations.
Ces actuateur sont connecté en série et sont modulables. Ceci permet la reconfiguration du système 6-DDL en un système 3-DDL




La petite taille de ce système en fait le candidat idéal pour l'utilisation dans le cadre du développement d'un algorithme d'évitement de collision.





\section{Cinématique directe}
\section{Cinématique inverse}
\section{Matrice Jacobienne}
\chapter{Environnement de développement, robot 3DDL}     % numéroté

\section{Descriptioon de l'infrastructure déjà présente}
\section{Définition des objectifs}S
\section{Description de l'infrastructure}
\chapter{Architecture du robot 6DDL}     % numéroté

\section{Description des charactéristiques physiques}
\section{Cinématique directe}
\section{Cinématique inverse}
\section{Matrice Jacobienne}

\chapter{Environnement de développement, robot 6DDL}     % numéroté

\section{Descriptioon de l'infrastructure déjà présente}
\section{Définition des objectifs}
\section{Description de l'infrastructure}
\chapter{Algorithme d'évitement de collisions}     % numéroté

\section{Définition du probléme}
\section{Objectifs}
\section{Description de l'architecture de l'alogrithme}
\section{Gestion des limitations}
\subsection{Limitations survenant à l'effecteur}
\subsection{Limitations survenant ailleur sur le robot}
\section{Représentation géométrique des limitations}
\subsection{Cas à 3 dimensions}
\subsection{Cas à plus de 3 dimensions}

\chapter*{Conclusion}         % ne pas numéroter
\phantomsection\addcontentsline{toc}{chapter}{Conclusion} % dans TdM

Une thèse ou un mémoire devrait normalement se terminer par une
conclusion, placée avant les annexes, le cas échéant. Celle-ci est
traitée comme un chapitre normal, sauf qu'elle n'est pas numérotée.
            % conclusion

\appendix                       % annexes le cas échéant

\chapter{Titre de l'annexe}     % numérotée

Texte de l'annexe.
                % annexe A

\bibliography{}                 % production de la bibliographie

\end{document}
