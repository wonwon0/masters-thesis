%% GABARIT POUR THÈSE SUR MESURE
%%
%% Consulter la documentation de la classe ulthese pour une
%% description détaillée de la classe, de ce gabarit et des options
%% disponibles.
%%
%% [Ne pas hésiter à supprimer les commentaires après les avoir lus.]
%%
%% Déclaration de la classe avec le type de grade
%%   [l'un de LLD, DMus, DPsy, DThP, PhD]
%% et les langues les plus courantes. Le français sera la langue par
%% défaut du document.
\documentclass[PhD,english,french]{ulthese}
  %% Encodage utilisé pour les caractères accentués dans les fichiers
  %% source du document. Les gabarits sont encodés en UTF-8. Inutile
  %% avec XeLaTeX, qui gère Unicode nativement.
  \ifxetex\else \usepackage[utf8]{inputenc} \fi

  %% Charger ici les autres paquetages nécessaires pour le document.
  %% Quelques exemples; décommenter au besoin.
  %\usepackage{amsmath}       % recommandé pour les mathématiques
  %\usepackage{ncccomma}      % gestion de la virgule dans les nombres

  %% Utilisation d'une autre police de caractères pour le document.
  %% - Sous LaTeX
  %\usepackage{mathpazo}      % texte et mathématiques en Palatino
  %\usepackage{mathptmx}      % texte et mathématiques en Times
  %% - Sous XeLaTeX
  %\setmainfont{TeX Gyre Pagella}      % texte en Pagella (Palatino)
  %\setmathfont{TeX Gyre Pagella Math} % mathématiques en Pagella (Palatino)
  %\setmainfont{TeX Gyre Termes}       % texte en Termes (Times)
  %\setmathfont{TeX Gyre Termes Math}  % mathématiques en Termes (Times)

  %% Gestion des hyperliens dans le document. S'assurer que hyperref
  %% est le dernier paquetage chargé.
  \usepackage{hyperref}
  \hypersetup{colorlinks,allcolors=ULlinkcolor}

  %% Options de mise en forme du mode français de babel. Consulter la
  %% documentation du paquetage babel pour les options disponibles.
  %% Désactiver (effacer ou mettre en commentaire) si l'option
  %% 'nobabel' est spécifiée au chargement de la classe.
  \frenchbsetup{%
    StandardItemizeEnv=true,       % format standard des listes
    ThinSpaceInFrenchNumbers=true, % espace fine dans les nombres
    og=«, fg=»                     % caractères « et » sont les guillemets
  }

  %% Style de la bibliographie.
 \bibliographystyle{}

  %% Déclarations des pages de titre. Remplacer les éléments entre < >.
  %% Supprimer les caractères < >. Couper un long titre ou un long
  %% sous-titre manuellement avec \\.
  \titre{<Titre principal>}
  % \titre{Ceci est un exemple de long titre \\
  %   avec saut de ligne manuel}
  % \soustitre{Sous-titre le cas échéant}
  % \soustitre{Ceci est un exemple de long sous-titre \\
  %   avec saut de ligne manuel}
  \auteur{<Prénom Nom>}
  \annee{<20xx>}
  \programme{Doctorat sur mesure en <discipline> <-- majeure, s'il y a lieu>}
  \direction{<Prénom Nom>, <directeur ou directrice> de recherche}
  % \codirection{<Prénom Nom>, <codirecteur ou codirectrice> de recherche}
  % \codirection{<Prénom Nom>, <codirecteur ou codirectrice> de recherche \\
  %              <Prénom Nom>, <codirecteur ou codirectrice> de recherche}

\begin{document}

\frontmatter                    % pages liminaires

\pagestitre                     % production des pages de titre

\chapter*{Résumé}                      % ne pas numéroter
\phantomsection\addcontentsline{toc}{chapter}{Résumé} % inclure dans TdM

\begin{otherlanguage*}{french}
  Texte du résumé en français.
\end{otherlanguage*}
                % résumé français
\chapter*{Abstract}                      % ne pas numéroter
\phantomsection\addcontentsline{toc}{chapter}{Abstract} % inclure dans TdM

\begin{otherlanguage*}{english}
  Text of English abstract.
\end{otherlanguage*}
              % résumé anglais
\cleardoublepage

\tableofcontents                % production de la TdM
\cleardoublepage

\listoftables                   % production de la liste des tableaux
\cleardoublepage

\listoffigures                  % production de la liste des figures
\cleardoublepage

\dedicace{Dédicace si désiré}
\cleardoublepage

\epigraphe{Texte de l'épigraphe}{Source ou auteur}
\cleardoublepage

\chapter*{Remerciements}         % ne pas numéroter
\phantomsection\addcontentsline{toc}{chapter}{Remerciements} % inclure dans TdM

Texte des remerciements en prose.
         % remerciements
\chapter*{Avant-propos}         % ne pas numéroter
\phantomsection\addcontentsline{toc}{chapter}{Avant-propos} % inclure dans TdM

L'avant-propos est surtout nécessaire pour une thèse par article.
           % avant-propos

\mainmatter                     % corps du document

\chapter*{Introduction}         % ne pas numéroter
\phantomsection\addcontentsline{toc}{chapter}{Introduction} % inclure dans TdM

Une thèse ou un mémoire devrait normalement débuter par une
introduction. Celle-ci est traitée comme un chapitre normal, sauf
qu'elle n'est pas numérotée.
          % introduction
\chapter{Architecture du robot 3DDL}     % numéroté

Texte du chapitre.
             % chapitre 1
\chapter{Développement de l'environnement de développement, robot 3DDL}     % numéroté

Texte du chapitre.
             % chapitre 2, etc.
\chapter*{Conclusion}         % ne pas numéroter
\phantomsection\addcontentsline{toc}{chapter}{Conclusion} % dans TdM

Une thèse ou un mémoire devrait normalement se terminer par une
conclusion, placée avant les annexes, le cas échéant. Celle-ci est
traitée comme un chapitre normal, sauf qu'elle n'est pas numérotée.
            % conclusion

\appendix                       % annexes le cas échéant

\chapter{Titre de l'annexe}     % numérotée

Texte de l'annexe.
                % annexe A

\bibliography{}                 % production de la bibliographie

\end{document}
