\chapter{Description du robot 3 degrès de liberté}     % numéroté

Afin de réaliser l'implémentation de l'algoritme d'évitement de collisions sur un robot 3 degrès de liberté, une base robotique à du être sélectionnée.
Le robot utilisé est un robot Jaco, de l'entreprise Kinova, modifié pour les besoins du projet.
Le présent chapitre traite alors des différentes charactéristiques physiques du robot modifié, des modèles de cinématique directe et inverse ainsi que du calcul de la matrice Jacobienne du système.

\section{Description des charactéristiques physiques du robot}
Le robot Jaco, de Kinova, est un robot 6 degrès de liberté utilisé principalement dans le domaine de la santé où il est utilisé, entre autre, pour augmenter l'autonomie de patients ayant perdu le plein usage de leur bras.
Le robot, dans sa configuration normale, pèse 5.2 Kg, peut soulever une charge maximale de 1.6 Kg de manière continue et posède une porté de 90 cm.
L'achitecture du robot Jaco comprend 6 membrures creuses en fibre de carbone ainsi qu'un effecteur comprenant 3 doights ayant 2 articulations.

La figure \ref{fig:jaco_6dof} présente l'architecture du robot dans ça configuration originale.
\begin{figure}
 \begin{center}
  \begin{tabular}{c}
    \includegraphics[trim=0cm 0cm 0cm 0cm, scale=0.17]{"figures/jaco_6dof"}
  \end{tabular}
 \end{center}
\caption{Représentation de l'architecture du robot Jaco 6DOF}
 \label{fig:jaco_6dof}
\end{figure}

Deux modèles différents d'actuateurs sont présents dans le robot, soit le modèle K-75+ pour les 3 première articulations et le modèle K-58 pour les 3 autres articulations.
Ces actuateur sont connecté en série et sont modulables. Ceci permet la reconfiguration du système 6-DDL en un système 3-DDL




La petite taille de ce système en fait le candidat idéal pour l'utilisation dans le cadre du développement d'un algorithme d'évitement de collision.





\section{Cinématique directe}
\section{Cinématique inverse}
\section{Matrice Jacobienne}